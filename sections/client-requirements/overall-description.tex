\section{Overall description}

%%%%%%%%%%%%%%%%%%%%%%%%%%%%%%%%%%%%%%%%%%%%%%%%%%%%%%%%%%%%%%%%%%%%%%%%%%%%%%%%
% This section of the SRS should describe the general factors that affect the product and its requirements.
% This section does not state specific requirements.
% Instead, it provides a background for those requirements, which are defined in detail in Section 3 of the SRS, and makes them easier to understand.
%%%%%%%%%%%%%%%%%%%%%%%%%%%%%%%%%%%%%%%%%%%%%%%%%%%%%%%%%%%%%%%%%%%%%%%%%%%%%%%%

\subsection{Product perspective}

%%%%%%%%%%%%%%%%%%%%%%%%%%%%%%%%%%%%%%%%%%%%%%%%%%%%%%%%%%%%%%%%%%%%%%%%%%%%%%%%
% This subsection of the SRS should put the product into perspective with other related products.
% If the product is independent and totally self-contained, it should be so stated here.
% If the SRS defines a product that is a component of a larger system, as frequently occurs, then this subsection should relate the requirements of that larger system to functionality of the software and should identify interfaces between that system and the software.
% A block diagram showing the major components of the larger system, interconnections, and external interfaces can be helpful.
% This subsection should also describe how the software operates inside various constraints.
% For example, these constraints could include
% 1. System interfaces;
% 2. User interfaces;
% 3. Hardware interfaces;
% 4. Software interfaces;
% 5. Communications interfaces;
% 6. Memory;
% 7. Operations;
% 8. Site adaptation requirements.
%%%%%%%%%%%%%%%%%%%%%%%%%%%%%%%%%%%%%%%%%%%%%%%%%%%%%%%%%%%%%%%%%%%%%%%%%%%%%%%%

\subsubsection{System interfaces}

%%%%%%%%%%%%%%%%%%%%%%%%%%%%%%%%%%%%%%%%%%%%%%%%%%%%%%%%%%%%%%%%%%%%%%%%%%%%%%%%
% This should list each system interface and identify the functionality of the software to accomplish the system requirement and the interface description to match the system.
%%%%%%%%%%%%%%%%%%%%%%%%%%%%%%%%%%%%%%%%%%%%%%%%%%%%%%%%%%%%%%%%%%%%%%%%%%%%%%%%

\subsubsection{User interfaces}

%%%%%%%%%%%%%%%%%%%%%%%%%%%%%%%%%%%%%%%%%%%%%%%%%%%%%%%%%%%%%%%%%%%%%%%%%%%%%%%%
% This should specify 
% 1. The logical characteristics of each interface between the software product and its users.
%    This includes those configuration characteristics (e.g., required screen formats, page or window layouts, content of any reports or menus, or availability of programmable function keys) necessary to accomplish the software requirements.
% 2. All the aspects of optimizing the interface with the person who must use the system.
%    This may simply comprise a list of do's and don'ts on how the system will appear to the user.
%    One example may be a requirement for the option of long or short error messages.
%    Like all others, these requirements should be verifiable, e.g., “a clerk typist grade 4 can do function X in Z min after 1 h of training” rather than “a typist can do function X.” (This may also be specified in the Software System Attributes under a section titled Ease of Use.)
%%%%%%%%%%%%%%%%%%%%%%%%%%%%%%%%%%%%%%%%%%%%%%%%%%%%%%%%%%%%%%%%%%%%%%%%%%%%%%%%

\subsubsection{Hardware interfaces}

%%%%%%%%%%%%%%%%%%%%%%%%%%%%%%%%%%%%%%%%%%%%%%%%%%%%%%%%%%%%%%%%%%%%%%%%%%%%%%%%
% This should specify the logical characteristics of each interface between the software product and the hardware components of the system.
% This includes configuration characteristics (number of ports, instruction sets, etc.).
% It also covers such matters as what devices are to be supported, how they are to be supported, and protocols.
% For example, terminal support may specify full-screen support as opposed to line-by-line support.
%%%%%%%%%%%%%%%%%%%%%%%%%%%%%%%%%%%%%%%%%%%%%%%%%%%%%%%%%%%%%%%%%%%%%%%%%%%%%%%%

\subsubsection{Software interfaces}

%%%%%%%%%%%%%%%%%%%%%%%%%%%%%%%%%%%%%%%%%%%%%%%%%%%%%%%%%%%%%%%%%%%%%%%%%%%%%%%%
% This should specify the use of other required software products (e.g., a data management system, an operating system, or a mathematical package), and interfaces with other application systems (e.g., the linkage between an accounts receivable system and a general ledger system).
% For each required software product, the following should be provided:
% - Name;
% - Mnemonic;
% - Specification number;
% - Version number;
% - Source.
% For each interface, the following should be provided:
% - Discussion of the purpose of the interfacing software as related to this software product.
% -  Definition of the interface in terms of message content and format.
%    It is not necessary to detail any well-documented interface, but a reference to the document defining the interface is required.
%%%%%%%%%%%%%%%%%%%%%%%%%%%%%%%%%%%%%%%%%%%%%%%%%%%%%%%%%%%%%%%%%%%%%%%%%%%%%%%%

\subsubsection{Communications interfaces}

%%%%%%%%%%%%%%%%%%%%%%%%%%%%%%%%%%%%%%%%%%%%%%%%%%%%%%%%%%%%%%%%%%%%%%%%%%%%%%%%
% This should specify the various interfaces to communications such as local network protocols, etc.
%%%%%%%%%%%%%%%%%%%%%%%%%%%%%%%%%%%%%%%%%%%%%%%%%%%%%%%%%%%%%%%%%%%%%%%%%%%%%%%%

\subsubsection{Memory constraints}

%%%%%%%%%%%%%%%%%%%%%%%%%%%%%%%%%%%%%%%%%%%%%%%%%%%%%%%%%%%%%%%%%%%%%%%%%%%%%%%%
% This should specify any applicable characteristics and limits on primary and secondary memory.
%%%%%%%%%%%%%%%%%%%%%%%%%%%%%%%%%%%%%%%%%%%%%%%%%%%%%%%%%%%%%%%%%%%%%%%%%%%%%%%%

\subsubsection{Operations}

%%%%%%%%%%%%%%%%%%%%%%%%%%%%%%%%%%%%%%%%%%%%%%%%%%%%%%%%%%%%%%%%%%%%%%%%%%%%%%%%
% This should specify the normal and special operations required by the user such as
% 1. The various modes of operations in the user organization (e.g., user-initiated operations);
% 2. Periods of interactive operations and periods of unattended operations;
% 3. Data processing support functions;
% 4. Backup and recovery operations.
% This is sometimes specified as part of the User Interfaces section.
%%%%%%%%%%%%%%%%%%%%%%%%%%%%%%%%%%%%%%%%%%%%%%%%%%%%%%%%%%%%%%%%%%%%%%%%%%%%%%%%

\subsubsection{Site adaptation requirements}

%%%%%%%%%%%%%%%%%%%%%%%%%%%%%%%%%%%%%%%%%%%%%%%%%%%%%%%%%%%%%%%%%%%%%%%%%%%%%%%%
% This should
% 3. Define the requirements for any data or initialization sequences that are specific to a given site, mission, or operational mode (e.g., grid values, safety limits, etc.);
% 2. Specify the site or mission-related features that should be modified to adapt the software to a particular installation.
%%%%%%%%%%%%%%%%%%%%%%%%%%%%%%%%%%%%%%%%%%%%%%%%%%%%%%%%%%%%%%%%%%%%%%%%%%%%%%%%

\subsection{Product functions}

%%%%%%%%%%%%%%%%%%%%%%%%%%%%%%%%%%%%%%%%%%%%%%%%%%%%%%%%%%%%%%%%%%%%%%%%%%%%%%%%
% This subsection of the SRS should provide a summary of the major functions that the software will perform.
% For example, an SRS for an accounting program may use this part to address customer account maintenance, customer statement, and invoice preparation without mentioning the vast amount of detail that each of those functions requires.
% Sometimes the function summary that is necessary for this part can be taken directly from the section of the higher-level specification (if one exists) that allocates particular functions to the software product.
% Note that for the sake of clarity
% 1. The functions should be organized in a way that makes the list of functions understandable to the customer or to anyone else reading the document for the first time.
% 
% 2. Textual or graphical methods can be used to show the different functions and their relationships.
%    Such a diagram is not intended to show a design of a product, but simply shows the logical relationships among variables.
%%%%%%%%%%%%%%%%%%%%%%%%%%%%%%%%%%%%%%%%%%%%%%%%%%%%%%%%%%%%%%%%%%%%%%%%%%%%%%%%

\subsection{User characteristics}

%%%%%%%%%%%%%%%%%%%%%%%%%%%%%%%%%%%%%%%%%%%%%%%%%%%%%%%%%%%%%%%%%%%%%%%%%%%%%%%%
% This subsection of the SRS should describe those general characteristics of the intended users of the product including educational level, experience, and technical expertise.
% It should not be used to state specific requirements, but rather should provide the reasons why certain specific requirements are later specified in Section 3 of the SRS.
%%%%%%%%%%%%%%%%%%%%%%%%%%%%%%%%%%%%%%%%%%%%%%%%%%%%%%%%%%%%%%%%%%%%%%%%%%%%%%%%

\subsection{Constraints}

%%%%%%%%%%%%%%%%%%%%%%%%%%%%%%%%%%%%%%%%%%%%%%%%%%%%%%%%%%%%%%%%%%%%%%%%%%%%%%%%
% This subsection of the SRS should provide a general description of any other items that will limit the developer's options.
% These include
% 
% 1. Regulatory policies;
% 2. Hardware limitations (e.g., signal timing requirements);
% 3. Interfaces to other applications;
% 4. Parallel operation;
% 5. Audit functions;
% 6. Control functions;
% 7. Higher-order language requirements;
% 8. Signal handshake protocols (e.g., XON-XOFF, ACK-NACK);
% 9. Reliability requirements;
% 10. Criticality of the application;
% 11 Safety and security considerations.
%%%%%%%%%%%%%%%%%%%%%%%%%%%%%%%%%%%%%%%%%%%%%%%%%%%%%%%%%%%%%%%%%%%%%%%%%%%%%%%%

\subsection{Assumptions and dependencies}

%%%%%%%%%%%%%%%%%%%%%%%%%%%%%%%%%%%%%%%%%%%%%%%%%%%%%%%%%%%%%%%%%%%%%%%%%%%%%%%%
% This subsection of the SRS should list each of the factors that affect the requirements stated in the SRS.
% These factors are not design constraints on the software but are, rather, any changes to them that can affect the requirements in the SRS.
% For example, an assumption may be that a specific operating system will be available on the hardware designated for the software product.
% If, in fact, the operating system is not available, the SRS would then have to change accordingly.
%%%%%%%%%%%%%%%%%%%%%%%%%%%%%%%%%%%%%%%%%%%%%%%%%%%%%%%%%%%%%%%%%%%%%%%%%%%%%%%%

\subsection{Apportioning of requirements}

%%%%%%%%%%%%%%%%%%%%%%%%%%%%%%%%%%%%%%%%%%%%%%%%%%%%%%%%%%%%%%%%%%%%%%%%%%%%%%%%
% This subsection of the SRS should identify requirements that may be delayed until future versions of the system.
%%%%%%%%%%%%%%%%%%%%%%%%%%%%%%%%%%%%%%%%%%%%%%%%%%%%%%%%%%%%%%%%%%%%%%%%%%%%%%%%
