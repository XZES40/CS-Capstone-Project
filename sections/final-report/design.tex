\section{Project design}
%%%%%%%%%%%%%%%%%%%%%%%%%%%%%%%%%%%%%%%%%%%%%%%%%%%%%%%%%%%%%%%%%%%%%%%%%%%%%%%%
% Introduction.
% What the project design is.
%%%%%%%%%%%%%%%%%%%%%%%%%%%%%%%%%%%%%%%%%%%%%%%%%%%%%%%%%%%%%%%%%%%%%%%%%%%%%%%%

\subsection{Preliminary project design}
%%%%%%%%%%%%%%%%%%%%%%%%%%%%%%%%%%%%%%%%%%%%%%%%%%%%%%%%%%%%%%%%%%%%%%%%%%%%%%%%
% Your original design document
%%%%%%%%%%%%%%%%%%%%%%%%%%%%%%%%%%%%%%%%%%%%%%%%%%%%%%%%%%%%%%%%%%%%%%%%%%%%%%%%
\subsubsection{Introduction}

\paragraph{Purpose}

The purpose of this document is to outline the entirety of the design of the XZES40 application for the purposes of referencing during development and for communication with the project sponsor(s).

\paragraph{Scope}

The scope of this document is to outline in necessarily complex terms how the XZES40-Transformer application will be developed.

\paragraph{Development Time-line}

The following Gantt chart outlines the projected time-line for development.
This starts at the first week of January (winter-term week 1), and goes through project completion in June (end of spring term).

\begin{figure}
    \begin{PstGanttChart}[yunit=1,
                          xunit=1,
                          ChartUnitIntervalName=W,
                          ChartUnitBasicIntervalName=W,
                          TaskUnitIntervalValue=10,
                          TaskUnitType=W,
                          ChartStartInterval=1,
                          ChartShowIntervals]{20}{19}
        \psset{gradangle=180}
        \PstGanttTask[TaskStyle=Important,TaskInsideLabel={Alpha}]{0}{6}
        \PstGanttTask[TaskInsideLabel={Benchmark Data Collection}]{0}{3}
        \PstGanttTask[TaskInsideLabel={Core Functionality}]{0}{11}
        \PstGanttTask[TaskInsideLabel={Basic Transformation}]{0}{3}
        \PstGanttTask[TaskInsideLabel={Caching}]{3}{7}
        \PstGanttTask[TaskInsideLabel={Parallel Computation}]{3}{7}
        \PstGanttTask[TaskInsideLabel={Demo}]{5}{1}
        \PstGanttTask[TaskStyle=Important,TaskInsideLabel={Beta}]{6}{5}
        \PstGanttTask[TaskInsideLabel={CGI Script}]{7}{1}
        \PstGanttTask[TaskInsideLabel={Web Interface}]{8}{2}
        \PstGanttTask[TaskInsideLabel={Debian Package}]{9}{2}
        \PstGanttTask[TaskInsideLabel={Demo}]{10}{1}
        \PstGanttTask[TaskStyle=Important,TaskInsideLabel={Release}]{12}{7}
        \PstGanttTask[TaskInsideLabel={CLI Interface}]{12}{2}
        \PstGanttTask[TaskInsideLabel={RedHat Package}]{12}{6}
        \PstGanttTask[TaskInsideLabel={BSD Package}]{12}{6}
        \PstGanttTask[TaskInsideLabel={Windows Package}]{12}{6}
        \PstGanttTask[TaskInsideLabel={Demo}]{17}{2}
        \PstGanttTask[TaskStyle=NotImportant]{11}{1}
    \end{PstGanttChart}
    \caption{Development Gantt Chart Timeline.}
\end{figure}

\paragraph{Summary}

XZES40-Transformer is an application which transforms one XML with one XSLT document for the purposes of data transformation, like that of a spreadsheet program like Excel.
The uses for this rage widely from business to scientific to personal uses.
For most needs, an XML/XSLT document transformer needs to be able to handle a high volume of requests by a wide variety of clients.
To address these needs the XZES40 team is building an Open Source XML/XSLT document transformer with key optimizations built in to improve the document transformation process; helping businesses, institutions, and individuals get more done in a day.
\cite{xml-spec} \cite{xslt-spec} 

\paragraph{Issuing organization}

This document has been issued by Oregon State University and the Apache Software Foundation through the OSU 2016-2017 CS Capstone class.

\paragraph{Change history}
\begin{figure}
  \begin{tabular}{ | p{5cm} | p{5cm} |}
  \hline
    Revision & Date \\ \hline
    Working Draft & 2016-11-17 \\ \hline
  \end{tabular}
  \caption{Change History Table}
\end{figure}


\input{sections/final-report/design-body.tex}

\subsection{Changes}
%%%%%%%%%%%%%%%%%%%%%%%%%%%%%%%%%%%%%%%%%%%%%%%%%%%%%%%%%%%%%%%%%%%%%%%%%%%%%%%%
% A discussion of what had to change over the course the year.
%%%%%%%%%%%%%%%%%%%%%%%%%%%%%%%%%%%%%%%%%%%%%%%%%%%%%%%%%%%%%%%%%%%%%%%%%%%%%%%%

The design of the application changed in a few minor ways during the course of development.
Some of these changes were a narrowing of the design while others were entirely added or removed components.
This section outlines those changes and why they were made.

\subsubsection{Daemon}

The application daemon, the constant running process which processes incoming requests and returns a compiled document, was ill-planned early on in development.
The ill-definition came mostly from a lack of understanding of how the caching process would work; it was undetermined if we would depend on a truly in-memory caching system or depend on some other system, like Redis, to store serialized files.

Once our client made clear that we should use an in-memory cache we were then tasked with determining the best way to execute this idea.
We started by implementing UNIX sockets as a means of communication between our Python written CGI script and the C/C++ written daemon.
This was shortly converted into a localhost network socket for convenience.

After we got communication between the Web API script and daemon we were then tasked with crafting an even loop and preventing race conditions.
This touched every aspect of the C/C++ code, requiring a heavy re-write of the existing ``main'' daemon code as well as some components of the caching.
The jist of this even loop is:

\begin{enumerate}
  \item Setup networking connection with API script.
  \item Enter while loop.
  \item Enter for loop. Each loop here represents a thread being spawned.
  \begin{enumerate}
    \item Receive a request. Format of request is specified in the next section.
    \item Lock cache.
    \item Process request.
    \item Unlock cache.
    \item Wait for another request.
  \end{enumerate}
  \item After five requests have been processed, clean the threads and start the while loop again.
\end{enumerate}

\subsubsection{Web API}

The web API is used to access the application over an HTTP(S) connection.
It is processed as an Apache CGI script written in Python.

When a request comes it it is formatted as an HTTP Form with the following fields and values:

\begin{itemize}
  \item xml: Content of XML input file.
  \item xsl: Contnet of XSLT input file.
  \item params: A JSON formatted key:value pair of custom parameters (discussed further in the next subsection).
\end{itemize}

This is processed using the CGI script library.
The input files are hashed, the resulting hash is each file's name; the files are saved on disk to ``/tmp/xzes/...''.
The two files and their parameters are concatenated and hashed together, this is the job ID and filename of the output file.
A quick check is made to see if this combination of files and parameters has already been processed.
If it has, the previously processed (and already locally saved) file is returned.
If the output file does not already exist, the job continues.

The job request is formatted as ``job\_id,/path/to/input.xml,/path/to/input.xslt/path/to/output.xml,param1,`val1',param2,`val2',...`.
The serialized request is sent over a local network to the C++ daemon and processed.
Since both the API script and daemon are on the same machine the files are opened directly from disk and the output file is written to the same directory.

When the request is over the daemon responds with a string ``job\_id,/path/to/output.xml,errors if any''.
The connection is then closed and the API send the file back to the user that made the request.

\subsubsection{Custom parameter and file passing}

One \textit{very} late addition to the application was custom parameter and additional file passing to the application.
The former is when a user sends a key:value pair to the transformer, each instance of the XML variable ``key'' is replaced with ``value'', usually being replaced with pre-defined default.
The latter is when a user sends an additional(s) file to the transformer which are required for the transformation to take place; these can be thought of as header files.

The parameter passing was defined in the above section.
Essentially the parameters are parsed by the web form (discussed below), serialized in JSON, and sent to the API to be decoded in python.

The file passing was not designed as it was never feasible to add this feature to the application.

\subsubsection{Interface (web)}

The web interface is written in stock HTML, CSS, JavaScript, and jQuery.

\begin{itemize}
  \item The page is a single form consisting of two fields, one to upload the XML file and the other to upload the XSL file, an ``add parameter'' button, and a ``submit form'' button.
  \item That buttons adds a field for the key:value pair passed to the XML transformer.
  \item When the form is ``submitted'' jQuery processes the form, serializes the parameters into JSON, and sends it asynchronously to the API endpoint ``/cgi-bin/''.
  \item If the transformation is a success, a button prompting the user to download the file and another to preview the file in the browser appear in the page.
  \item If the transformation is not a success an error message passed from the transformer is displayed instead.
\end{itemize}

\subsubsection{Interface (CLI)}

The CLI interface was never developed in favor of the web interface.

As it turns out the web interface was only marginally more difficult to develop than the CLI.

A CLI used for testing does exist in the source repository.

\subsubsection{Installation packages}

In addition to the CLI, the administrator installation package was never pursued.

Bash scripts were used to provision a development VM, these may be used to \textit{create} a development package, but one was not created for either Debian nor the other [stretch goal] targets.

This requirement was not explicitly dropped, just not pursued because of development constraints.

\subsubsection{OS Agnostic C/C++ API}

As development efforts were stretched thin, the dream of writing an OS agnostic API for system calls was not pursued as well.

This requirement was not explicitly dropped, just not pursued because of development constraints.
