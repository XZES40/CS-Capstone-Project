\subsubsection{Overall description}

%%%%%%%%%%%%%%%%%%%%%%%%%%%%%%%%%%%%%%%%%%%%%%%%%%%%%%%%%%%%%%%%%%%%%%%%%%%%%%%%
% This section of the SRS should describe the general factors that affect the product and its requirements.
% This section does not state specific requirements.
% Instead, it provides a background for those requirements, which are defined in detail in Section 3 of the SRS, and makes them easier to understand.
%%%%%%%%%%%%%%%%%%%%%%%%%%%%%%%%%%%%%%%%%%%%%%%%%%%%%%%%%%%%%%%%%%%%%%%%%%%%%%%%

The following sections of this document outline the factors that affect the creation of XZES40-Transformer at a high-level.

\paragraph{Product perspective}
%%%%%%%%%%%%%%%%%%%%%%%%%%%%%%%%%%%%%%%%%%%%%%%%%%%%%%%%%%%%%%%%%%%%%%%%%%%%%%%%
% This paragraph of the SRS should put the product into perspective with other related products.
% If the product is independent and totally self-contained, it should be so stated here.
% If the SRS defines a product that is a component of a larger system, as frequently occurs,
% then this paragraph should relate the requirements of that larger system to functionality of the software and should identify interfaces between that system and the software.
% A block diagram showing the major components of the larger system, interconnections, and external interfaces can be helpful.
% This paragraph should also describe how the software operates inside various constraints.
% For example, these constraints could include
% 1. System interfaces;
% 2. User interfaces;
% 3. Hardware interfaces;
% 4. Software interfaces;
% 5. Communications interfaces;
% 6. Memory;
% 7. Operations;
% 8. Site adaptation requirements.
%%%%%%%%%%%%%%%%%%%%%%%%%%%%%%%%%%%%%%%%%%%%%%%%%%%%%%%%%%%%%%%%%%%%%%%%%%%%%%%%

\textbf{System interfaces}
%%%%%%%%%%%%%%%%%%%%%%%%%%%%%%%%%%%%%%%%%%%%%%%%%%%%%%%%%%%%%%%%%%%%%%%%%%%%%%%%
% This should list each system interface and identify the functionality of the software to accomplish the system requirement and the interface description to match the system.
%%%%%%%%%%%%%%%%%%%%%%%%%%%%%%%%%%%%%%%%%%%%%%%%%%%%%%%%%%%%%%%%%%%%%%%%%%%%%%%%

The XZES40-Transformer application will interface with the outside world over the internet via the HTTP networking protocol.
The application will recieve HTTP POST requests to the application URI endpoint containing the documents to be transformed.
Once the transformation is completed the transformed document will be sent to the user for download.

In addition to the transformed file the application will respond with an HTTP \textbf{OK} status.
If an error occurs it will respond with a \textbf{SERVER ERROR} status and no file.

% \begin{enumerate}
%   \item Data base
%   \item User and manager
%   \item XML servers
% \end{enumerate}
%
% Our server has 3 layer of actor, first layer is user part, such as reader and manager.
% Second layer is XML servers, which contain XML parse and other XML transformation part.
% Third layer is our database where we store our data.
% User can connect to webpage and send request to XML servers,and then XML servers will search from database and query data out, after that XML server can compile to Unicode type file and give back to user.

\textbf{User interfaces}
%%%%%%%%%%%%%%%%%%%%%%%%%%%%%%%%%%%%%%%%%%%%%%%%%%%%%%%%%%%%%%%%%%%%%%%%%%%%%%%%
% This should specify 
% 1. The logical characteristics of each interface between the software product and its users.
%    This includes those configuration characteristics (e.g., required screen formats, page or window layouts, 
%    content of any reports or menus, or availability of programmable function keys) necessary to accomplish the software requirements.
% 2. All the aspects of optimizing the interface with the person who must use the system.
%    This may simply comprise a list of do's and don'ts on how the system will appear to the user.
%    One example may be a requirement for the option of long or short error messages.
%    Like all others, these requirements should be verifiable, e.g., “a clerk typist grade 4 can do function X in Z min after 1 h of training” rather than “a typist can do function X.” (This may also be specified in the Software System Attributes under a section titled Ease of Use.)
%%%%%%%%%%%%%%%%%%%%%%%%%%%%%%%%%%%%%%%%%%%%%%%%%%%%%%%%%%%%%%%%%%%%%%%%%%%%%%%%

XZES40 will have two user interfaces which will access it's functionality over the internet.

\begin{itemize}
    \item {
      A Website to access XZES40-Transformer via a web-browser.
      This interface will be called ``XZES40-Web''.
    }
    \item {
      A CLI to access XZES40-Transformer via a terminal interface.
      This interface will be called ``XZES40-CLI''.
    }
\end{itemize}

Both interfaces will not perform local document transformation.
They will instead access the transformation service over the internet, making the HTTP API convenient to use.

% Our application requires at least 800x600 resolution and 256 color monitor, and standard keyboard, and mouse is option.
% Managers can control database in their experience after see the GUI of transfer management application.
% User just need click webpage request, and other things is working behind the webpage.

\textbf{Hardware interfaces}
%%%%%%%%%%%%%%%%%%%%%%%%%%%%%%%%%%%%%%%%%%%%%%%%%%%%%%%%%%%%%%%%%%%%%%%%%%%%%%%%
% This should specify the logical characteristics of each interface between the software product and the hardware components of the system.
% This includes configuration characteristics (number of ports, instruction sets, etc.).
% It also covers such matters as what devices are to be supported, how they are to be supported, and protocols.
% For example, terminal support may specify full-screen support as opposed to line-by-line support.
%%%%%%%%%%%%%%%%%%%%%%%%%%%%%%%%%%%%%%%%%%%%%%%%%%%%%%%%%%%%%%%%%%%%%%%%%%%%%%%%

XZES40-Transformer will not have any direct physical interfaces as it is meant to be interacted with over HTTP or HTTPS.
Any computer with an internet connection, monitor, input methods, and web-browser will be able to access XZES40-Transformer via the web interface.
Any computer with an internet connection, monitor, input methods, and which has the CLI installed will be able to access XZES40-Transformer via the CLI interface.

The application will be targeted to run on a Debian Linux 8 (``Jessie'') X86\_64 CPU architecture server.
This machine should have one port open for communicating over HTTP (port 80) and one for HTTPS (port 443) if that is configured.

% We will support windows, Unix based operation  system, iOS and android.
% We will create API for our application, so we can easy to move application to any platform.
% GUI is required rather than line by line terminal control.

\textbf{Software interfaces}
%%%%%%%%%%%%%%%%%%%%%%%%%%%%%%%%%%%%%%%%%%%%%%%%%%%%%%%%%%%%%%%%%%%%%%%%%%%%%%%%
% This should specify the use of other required software products (e.g., a data management system, an operating system, or a mathematical package),
% and interfaces with other application systems (e.g., the linkage between an accounts receivable system and a general ledger system).
% For each required software product, the following should be provided:
% - Name;
% - Mnemonic;
% - Specification number;
% - Version number;
% - Source.
% For each interface, the following should be provided:
% - Discussion of the purpose of the interfacing software as related to this software product.
% - Definition of the interface in terms of message content and format.
%   It is not necessary to detail any well-documented interface, but a reference to the document defining the interface is required.
%%%%%%%%%%%%%%%%%%%%%%%%%%%%%%%%%%%%%%%%%%%%%%%%%%%%%%%%%%%%%%%%%%%%%%%%%%%%%%%%

Below is a list of software required for the XZES-40 (on the host and on remote systems).

\begin{description}
  \item {
    \begin{description}
      \item Name: Xerces-C++ XML Parser
      \item Mnemonic: Xerces-C
      \item Specification Number: XML 1.0 specification is implemented.
      \item Version Number: 3.1.4 (Recent)
      \item Source: \url{http://xerces.apache.org/xerces-c/}
    \end{description}
  }
  \item {
    \begin{description}
      \item Name: Xalan-C++ XSLT Processor 
      \item Mnemonic: Xalan-C
      \item Specification Number: XML 1.0 specification is implemented.
      \item Version Number: 1.10 (Recent)
      \item Source: \url{http://xalan.apache.org/xalan-c/}
    \end{description}
  }
  \item {
    \begin{description}
      \item Name: International Components for Unicode
      \item Mnemonic: ICU
      \item Specification Number: Unicode 9.0
      \item Version Number: ICU 58
      \item Source: \url{http://site.icu-project.org/download/58}
    \end{description}
  }
   \item {
     \begin{description}
       \item Name: Apache CGI Processing
       \item Mnemonic: Apache CGI
       \item Version Number: Apache 2
       \item Source: \url{http://apache.org}
     \end{description}
  }
%  \item {
%     \begin{description}
%       \item Name: Python2
%       \item Mnemonic: Python
%       \item Version Number: 2.7+
%       \item Source: \url{http://python.org}
%     \end{description}
%   }
\end{description}


% SQL database is required for our application.
% Version of SQL is newer than SQL2003
% UNIX or Linux based on operation system is required, such as mac OS, debin, center OS.
% All of operation system should be newest version.

%GUI support package??

\textbf{Communications interfaces}
%%%%%%%%%%%%%%%%%%%%%%%%%%%%%%%%%%%%%%%%%%%%%%%%%%%%%%%%%%%%%%%%%%%%%%%%%%%%%%%%
% This should specify the various interfaces to communications such as local network protocols, etc.
%%%%%%%%%%%%%%%%%%%%%%%%%%%%%%%%%%%%%%%%%%%%%%%%%%%%%%%%%%%%%%%%%%%%%%%%%%%%%%%%

An internet connection between the host and client is required for use of XZES40-Transformer.
The host and client will be communicating over HTTP or HTTPS through the web interface or CLI interface.
The application may be deployed behind a firewall.

Administrators may deploy multiple instance of the application, however additional instances will not be designed to communicate together, so they will each act autonomously.

% The data is exchanged between the catalog system and the Enterprise Buyer system via a web browser (HTTP or HTTPS).
% After the data has been sent in XML format to the Enterprise Buyer application server, depending on the OCL version, 
% an ABAP-XSL transformation is carried out or the SAP Business Connector is used up map the XML file in an internal representation.
% The following graphic illustrates the data exchange between the external catalog system and Enterprise Buyer.

\textbf{Memory constraints}
%%%%%%%%%%%%%%%%%%%%%%%%%%%%%%%%%%%%%%%%%%%%%%%%%%%%%%%%%%%%%%%%%%%%%%%%%%%%%%%%
% This should specify any applicable characteristics and limits on primary and secondary memory.
%%%%%%%%%%%%%%%%%%%%%%%%%%%%%%%%%%%%%%%%%%%%%%%%%%%%%%%%%%%%%%%%%%%%%%%%%%%%%%%%

XZES40-Transformer will depend heavily on an internal caching system, so an appropriate amount of memory should be dedicated to the application.
The specific amount of memory will depend on how much an instance of the application is expected to be used, however
a minimum of 4GiB should be dedicated to the machine it is running on.
That said, the more the merrier.

\textbf{Operations}
%%%%%%%%%%%%%%%%%%%%%%%%%%%%%%%%%%%%%%%%%%%%%%%%%%%%%%%%%%%%%%%%%%%%%%%%%%%%%%%%
% This should specify the normal and special operations required by the user such as
% 1. The various modes of operations in the user organization (e.g., user-initiated operations);
% 2. Periods of interactive operations and periods of unattended operations;
% 3. Data processing support functions;
% 4. Backup and recovery operations.
% This is sometimes specified as part of the User Interfaces section.
%%%%%%%%%%%%%%%%%%%%%%%%%%%%%%%%%%%%%%%%%%%%%%%%%%%%%%%%%%%%%%%%%%%%%%%%%%%%%%%%

XZES40-Transformer will initially target the Debian Jessie operating system.
To install the application a system administrator will acquire a Debian installation package (\inlinecode{xzes40-transformer.deb}), run the installation file, and begin the newly installed service.

The steps will roughly be as follows:
\begin{lstlisting}
Download xzes40-transformer.deb
$ wget http://example.com/xzes40-transformer.deb

Install the package
$ dpkg -i xzes40-transformer.deb

Enable the xzes40-transformer Systemd service
$ systemctl enable xzes40-transformer

Start the Systemd service
$ systemctl start xzes40-transformer
\end{lstlisting}

Installation on an additional Debian system would require the same procedure as listed above.

Installation on a non-Debian operating system will require a system-specific installation file, which we will create.
The non-Debian systems we will include:
\begin{itemize}
  \item Windows 7+
  \item MacOS
  \item FreeBSD
  \item RedHat Enterprise Linux
\end{itemize}

Once installed and setup, XZES40-Transformer will require minimal user-interaction by the system administrator.
The application will run as a daemon on the host system.
If a fatal error occurs the daemon will restart.

While users are not interacting with the application it will idle in the background.
During periods of intense use the application will manage it's own resources to avoid breaking.

As the application does not store ephemeral data, there will not be a need for data backup nor data restoration.

\textbf{Site adaptation requirements}
%%%%%%%%%%%%%%%%%%%%%%%%%%%%%%%%%%%%%%%%%%%%%%%%%%%%%%%%%%%%%%%%%%%%%%%%%%%%%%%%
% This should
% 3. Define the requirements for any data or initialization sequences that are specific to a given site, mission, 
% or operational mode (e.g., grid values, safety limits, etc.);
% 2. Specify the site or mission-related features that should be modified to adapt the software to a particular installation.
%%%%%%%%%%%%%%%%%%%%%%%%%%%%%%%%%%%%%%%%%%%%%%%%%%%%%%%%%%%%%%%%%%%%%%%%%%%%%%%%

XZES40-Transformer will use Apache to manage web-requests.
If users want to setup secure communication over HTTPS they will need to do this manually using Apache.

The application will include a configuration file to specify resource limits, and other relevant information.
This file should be tailored to a given user's installation and needs.

In addition to the configuration file, the user may configure the daemon through the daemon manager (\inlinecode{systemd} for instance) to set hard-limits on the application's resource usage.

% XML is the most important and commonly used form of the most commonly used data exchange representation on the grid today.
% XML is a subset of SGML that describes various types of data in a structured way.
% It allows document creators to create new tags to better describe the data.
% XML can describe almost all areas of data.
% It uses strict nested tokens to represent data, and is particularly well suited for multi-site data exchange environments on the Internet.
% XML itself is extensible and only defines standard syntax.
% XML is a language for creating industry vocabularies and applications whose basic syntax for the file is defined by the XML schema defined by the W3C Create File.
% In the grid environment, because of the XML file structure and readability, XML data often as documents or process data to the form of cooperation in circulation, it also needs to use encryption and signature to ensure that XML-based data exchange activities in the information safety.
% The security of XML language is the foundation of information exchange on grid.
% To protect the security of XML data exchange, the International Organization for Standardization W3C proposed a series of XML security services, the new standard for XML as a data exchange carrier to provide security applications.
% These standards include: XML Encryption, XML Sigllature, XML Key Management Specification (XKMS), XML Access % Control Markup Language (XACML), etc.

\paragraph{Product functions}
%%%%%%%%%%%%%%%%%%%%%%%%%%%%%%%%%%%%%%%%%%%%%%%%%%%%%%%%%%%%%%%%%%%%%%%%%%%%%%%%
% This paragraph of the SRS should provide a summary of the major functions that the software will perform.
% For example, an SRS for an accounting program may use this part to address customer account maintenance, customer statement,
% and invoice preparation without mentioning the vast amount of detail that each of those functions requires.
% Sometimes the function summary that is necessary for this part can be taken directly from the section of 
% the higher-level specification (if one exists) that allocates particular functions to the software product.
% Note that for the sake of clarity
% 1. The functions should be organized in a way that makes the list of functions understandable to the customer
%    or to anyone else reading the document for the first time.
% 
% 2. Textual or graphical methods can be used to show the different functions and their relationships.
%    Such a diagram is not intended to show a design of a product, but simply shows the logical relationships among variables.
%%%%%%%%%%%%%%%%%%%%%%%%%%%%%%%%%%%%%%%%%%%%%%%%%%%%%%%%%%%%%%%%%%%%%%%%%%%%%%%%

XZES40-Transformer will perform one function: XML/XSLT document transformation.
Given one XML and one XSLT documents it will return a transformed XML document.

This functionality will be remotely accessible via an HTTP web-page and CLI interface.

% Our XML server are common XML documents transfer servers.
% The big different between normal XML server and our XML server is that we create cache for our server, so this prevent xml server compile each time which is waste system source.
% So, user can easily click webpage request and receive file that they want.
% Our server just quickly grab data from database, compile files, and send back to user.
% we also create server management application for manager, and it is graphic interface application, so people easy to use.

\paragraph{User characteristics}
%%%%%%%%%%%%%%%%%%%%%%%%%%%%%%%%%%%%%%%%%%%%%%%%%%%%%%%%%%%%%%%%%%%%%%%%%%%%%%%%
% This paragraph of the SRS should describe those general characteristics of the intended 
% users of the product including educational level, experience, and technical expertise.
% It should not be used to state specific requirements, but rather should provide the reasons 
% why certain specific requirements are later specified in Section 3 of the SRS.
%%%%%%%%%%%%%%%%%%%%%%%%%%%%%%%%%%%%%%%%%%%%%%%%%%%%%%%%%%%%%%%%%%%%%%%%%%%%%%%%

The \textbf{user} of our application is expected to have common web-interface knowledge (e.g., they should know how to navigate a website, upload a file, and download a file).

% User should understand the basically computer knowledge.
% They better be above high school educational level, 
% and they should have experience with windows application interface.
% Manager should understand C or C++ programming language, if manager want know how software working.

\paragraph{Constraints}
%%%%%%%%%%%%%%%%%%%%%%%%%%%%%%%%%%%%%%%%%%%%%%%%%%%%%%%%%%%%%%%%%%%%%%%%%%%%%%%%
% This paragraph of the SRS should provide a general description of any other items that will limit the developer's options.
% These include
% 
% 1. Regulatory policies;
% 2. Hardware limitations (e.g., signal timing requirements);
% 3. Interfaces to other applications;
% 4. Parallel operation;
% 5. Audit functions;
% 6. Control functions;
% 7. Higher-order language requirements;
% 8. Signal handshake protocols (e.g., XON-XOFF, ACK-NACK);
% 9. Reliability requirements;
% 10. Criticality of the application;
% 11 Safety and security considerations.
%%%%%%%%%%%%%%%%%%%%%%%%%%%%%%%%%%%%%%%%%%%%%%%%%%%%%%%%%%%%%%%%%%%%%%%%%%%%%%%%

XZES40-Transformer will be subject to the following limitations:

\begin{itemize}
  \item The code must be licensed under Apache 2.0.
  \item It must handle memory limitations gracefully.
  \item It must restart if a fatal error occurs.
  \item It must run on Debian 8.
  \item It must be accessible over a network.
  \item It must have an accessible interface.
\end{itemize}

% \begin{enumerate}
%  \item our application is open source, but copying code is forbid.
%   \item application should be running at larger severs.
%   \item C++ and C programming language is the first order language.
% \end{enumerate}

\paragraph{Assumptions and dependencies}
%%%%%%%%%%%%%%%%%%%%%%%%%%%%%%%%%%%%%%%%%%%%%%%%%%%%%%%%%%%%%%%%%%%%%%%%%%%%%%%%
% This paragraph of the SRS should list each of the factors that affect the requirements stated in the SRS.
% These factors are not design constraints on the software but are, rather, any changes to them that can affect the requirements in the SRS.
% For example, an assumption may be that a specific operating system will be available on the hardware designated for the software product.
% If, in fact, the operating system is not available, the SRS would then have to change accordingly.
%%%%%%%%%%%%%%%%%%%%%%%%%%%%%%%%%%%%%%%%%%%%%%%%%%%%%%%%%%%%%%%%%%%%%%%%%%%%%%%%

XZES40-Transformer will be written to interface with an OS agnostic API for any operating-system level operations (e.g., reading and writing from the cache).
The application will be portable to new operating systems by writing an OS-specific interface layer and compiling the binary for the given target platform (e.g., Windows or FreeBSD).

XZES40-Transformer will assume that the relevant libraries and languages listed under Software Interfaces are already installed.
The installation package we create will resolve these dependencies if they are not already installed with the correct version.

% Our application is working with a debin system on Intel servers.
% If debin is not installed, we can take other Linux operation system.
% If SQL database is not working, we also can take MySQL or SQL MS Access.
% If GUI of our application is broken, user can be using command line to access our application.

\paragraph{Apportioning of requirements}
%%%%%%%%%%%%%%%%%%%%%%%%%%%%%%%%%%%%%%%%%%%%%%%%%%%%%%%%%%%%%%%%%%%%%%%%%%%%%%%%
% This paragraph of the SRS should identify requirements that may be delayed until future versions of the system.
%%%%%%%%%%%%%%%%%%%%%%%%%%%%%%%%%%%%%%%%%%%%%%%%%%%%%%%%%%%%%%%%%%%%%%%%%%%%%%%%

Development of the application, user interfaces, and installation packages will be carried out over a 19 weeks, split into three development cycles: Alpha, Beta, and Release.

The Gantt chart can be seen in the Appendix, Figure 4.

\textbf{Alpha}

During the Alpha phase of development we will collect benchmark data, create our basic transformation functionality, and begin work on Cache and Parallel computation optimizations.
By the end of week three the application will be able to accept two input documents and output a transformed document.
After the initial transformation functionality is complete work on optimizing this process will take place by adding caching and parallel processing to the transformation cycle.

\textbf{Beta}

During the beta phrase of development the XZES40 team will begin work on the HTTP API, the web interface, the Debian package, and further optimizations on the transformation process.
Work on the web interface is not possible without the CGI interface first being added, however most other development can take place in parallel.

\textbf{Release}

During the Release phrase we will work exclusively on stretch goals including the CLI interface as well as the RedHat, BSD, and Windows packages.
While these goals would be nice to achieve, we understand that there will probably be overflow from the Alpha and Beta phases of development, so we hope to complete all required deliverables well before the release deadline.


% XZES40-Transformer will be considered complete if the following goals are completed:
%
% \begin{enumerate}
%   \item {
%     \textbf{A working XML/XSLT document transformer}.
%     This will include the following requirements (in order):
%     \begin{enumerate}
%       \item \textbf{Document compilation}.
%       \item \textbf{Data transformation}.
%       \item \textbf{Data caching}.
%       \item \textbf{Parallel computation}.
%     \end{enumerate}
%   }
%   \item An \textbf{Apache CGI script for HTTP access} to the application.
%   \item The \textbf{website interface}.
% \end{enumerate}
% 
% Time permitting we will complete the following as ``stretch goals''.
% 
% \begin{enumerate}
%   \item {
%     \textbf{Debian installation package}.
%     \begin{enumerate}
%       \item \textbf{Daemon setup}.
%       \item \textbf{Dependency resolution}.
%     \end{enumerate}
%   }
%   \item \textbf{Command-line interface}.
%   \item {
%     \textbf{RedHat Enterprise Linux installation package}.
%     \begin{enumerate}
%       \item \textbf{API compatility changes}.
%       \item \textbf{Daemon setup}.
%       \item \textbf{Dependency resolution}.
%     \end{enumerate}
%   }
%   \item {
%     \textbf{Windows installation package}.
%     \begin{enumerate}
%       \item \textbf{API compatility changes}.
%       \item \textbf{Daemon setup}.
%       \item \textbf{Dependency resolution}.
%     \end{enumerate}
%   }
%   \item {
%     \textbf{MacOS installation package}.
%     \begin{enumerate}
%       \item \textbf{API compatility changes}.
%       \item \textbf{Daemon setup}.
%       \item \textbf{Dependency resolution}.
%     \end{enumerate}
%   }
% \end{enumerate}
