\section{Technology review}
%%%%%%%%%%%%%%%%%%%%%%%%%%%%%%%%%%%%%%%%%%%%%%%%%%%%%%%%%%%%%%%%%%%%%%%%%%%%%%%%
% Introduction; what a technology review is.
%%%%%%%%%%%%%%%%%%%%%%%%%%%%%%%%%%%%%%%%%%%%%%%%%%%%%%%%%%%%%%%%%%%%%%%%%%%%%%%%

In creating a piece of software it is best to look around, see what exists, and make an informed decisions about the tools you should use to carry out the project.
This section outlines those technologies investigated and the reason the tools chosen were used.

\subsection{Preliminary technology review}
%%%%%%%%%%%%%%%%%%%%%%%%%%%%%%%%%%%%%%%%%%%%%%%%%%%%%%%%%%%%%%%%%%%%%%%%%%%%%%%%
% Your tech review, in its original form.
%%%%%%%%%%%%%%%%%%%%%%%%%%%%%%%%%%%%%%%%%%%%%%%%%%%%%%%%%%%%%%%%%%%%%%%%%%%%%%%%

This is the preliminary technology review with minimal edits.
It includes the original references page for authenticity.

\input{sections/final-report/tech-review-introduction.tex}
% Section One: Research and Benchmarking
% Owned by Zixun Lu 
%\subsubsubsubsection{Research and Benchmarking}
%The first application necessary for XML data processing are XML parsers and XML validators. The key aim is to check correctness of the input data, i.e. their conformance to either W3C recommendations or respective XML schemes. Hence, the benchmarks usually involve sets of correct and incorrect data and the goal is to test whether the application under test recognizes them correctly.
%\paragraph{XML Conformance Test Suites}
%Binary tests contain a set of documents of one of the following categories: valid documents, invalid documents, non-well-formed documents, well-formed errors tied to external entity and documents with optional errors. Depending on the category, the tested parser must either accept or reject the document correctly (therefore, the tests are called binary). The expected behavior naturally differs if the tested parser is validating or non-validating. On the other hand, the output tests enable to test whether the respective applications report information as required by the recommendation. Again, validating processors are required to report more information than non-validating ones.
%\paragraph{}

%\paragraph{}
\subsubsection{Research and Benchmarking}

\paragraph{Options}

\paragraph{Goals for use in design}

Our team will do the research and benchmarking to grantee that our application is fast enough.

\paragraph{Criteria being evaluated}

We will test some number of requests against a comparable to find which transformers use the time less.
Throughout development we will put our application through the same paces and compare which is faster.

\paragraph{Comparison breakdown}

\begin{center}
  \begin{tabular}{ | l | p{10cm} |}
    \hline
    Technology & Description  \\ \hline

    Xalan CLI \cite{Xalan-C} &
    \begin{itemize}
      \item Xalan-C++ uses Xerces-C++ to parse XML documents and XSL stylesheets.
      \item The project provides an open source CLI program to test the project libraries.
      \item Free and Open Source
      \item It works on the Debian operation system. 
    \end{itemize}\\ \hline

    Altova \cite{Altova} &
    \begin{itemize}
      \item To meet industry demands for an ultra-fast processor.
      \item It offers powerful, flexible options for developers including cml, python.
      \item Superior error reporting capabilities include reporting of multiple errors, detailed error descriptions.
      \item It only works in the Windows operation system. 
    \end{itemize} \\ \hline

  \end{tabular}
\end{center}

\paragraph{Discussion}

Xerces is a simple CLI application developed by the Xerces project to test the library.
This is very similar to our program as it is open source, uses the same libraries, but lacks the caching we will implement.

RaptorXML is built from the ground up to be optimized for the latest standard and parallel computing environments.
It is proprietary tool which we may try to out-perform as a stretch goal, but to start with out application will not try to out-perform.

\paragraph{Selection}

We will compare our application to the Xalan-C CLI as it is the closest competitor to our application.

\input{sections/final-report/tech-review-document-parsing.tex}
% Section Two: Document Transformation
% Owned by Shuai Peng
\subsubsection{XML/XSLT Document Transformation}

\paragraph{Option}

There are three options for the XML/XSLT Document Transformation.
The first option is Xalan-C++, and the second option is Saxon C, the third option is Sablotron.

\paragraph{Goals for use in design}

The major function of XZES40-Transformer is XML/XSLT document transforming.
Apache foundation provide many ways to achieve this function.

\paragraph{Criteria being evaluated}

The XML/XSLT document transformation is our major function for XZES40-Transformer application.
We want this transformation compatible with good parse tools, and version of XSLT and XPath.

\paragraph{Comparison breakdown}

The first option is Xalan-C++, this technology is required by our client, and this technology is supported by Apache.
Xalan-C++ is an XSLT processor for transforming XML documents into HTML, text, or other XML document types.
Xalan-C++ contain the XercesC tools, so we don't need consider the parse tools.

The second option is Saxon C, which is Saxon company project.
It performs the same operations as Xalan-C++, but it supports different version of XSLT and Xpath.

The third option technology is Sablotron.
It used same version of XSLT and Xpath with Xalan-C++, but it designed to be as small program.

\begin{center}
    \begin{tabular}{ | l | p{10cm} |}
    \hline
    Technology & Description  \\ \hline
    Xalan-C++ \cite{xalan} &
    \begin{itemize}
      \item Xalan-C++ is open source project developed by Apache. It is implemented by XSLT version 1.0 and XPath version 1.0.
      \item Xalan-C++ uses Xerces-C++ to parse XML documents and XSL style sheets.
    \end{itemize}\\ \hline
    Saxon C \cite{Saxon_c} &
    \begin{itemize}
      \item Saxon C is open source project. It is implemented by XSLT 2.0/3.0 version and XPath version 2.0/3.0.
      \item Saxon C use different parse tools to handle the date.
    \end{itemize}\\ \hline
    Sablotron \cite{Sablotron_intro} &
    \begin{itemize}
      \item Sablotron is open source project by gingerall. It is implemented by XSLT 1.0 and Xpath 1.0.
	  \item Sablotron need extra parse tools to complete transformation.
    \end{itemize}\\ \hline
    \end{tabular}
\end{center}

\paragraph{Discussion}

The Xalan-C++ is the most powerful and popular XML/XSLT document transformation library and is used in many propetary as well as open source tools.
It a robust implementation of the W3C recommendations for XSL Transformations (XSLT) and the XML Path Language (XPath).
The Xalan-C++ is continuing update, and it is good tools for transforming documents.
However, Xalan-C++ has poor structure API reference, and it makes developer hard to read and understand.
Thus, developer need spend time on doing research with API.

The Saxon C is a good alternative to Xalan-C++.
Saxon C can handle higher version of XSLT and Xpath, however implements different parse tools, so we need find out other technology.

The Sablotron does the same work as the Xalan-C++, but Sablotron is designed to be as compact and portable as possible.
The size of Sablotron is much small than Xalan-C.
However, Sablotron is old and has not been updated it since 2006.

\paragraph{The Best option}

We will choose Xalan-C++ as our solution technology.
The first reason is that Xalan-C is required by our client.
We want to using Xerces-C as our parse tools, so Xalan-C is the best choice.
Xalan-C++ is easy to use with clear example.

The second reason is that Sablotron is not stable.
Even they are open source project, but the community has forgetten it.
Saxon C is a good second option if client require different tools to complete XML/XSLT document transformation.

% Section Four: Document Caching
% Owned by Shuai Peng
\subsubsection{XML/XSLT Document Caching}

\paragraph{Option}

There are three technology options for cache.
The first option is storing cache into memory.
The second option is storing our cache in binary file on disk.
The third option is to create database to handle all of data.

\paragraph{Goals for use in design}

Caching is the "plus one" function of XZES40-Transformer application.
Other similar application compile files each time and this wastes a lot of time and resource.
We will create cache to solve this problem. 

\paragraph{Criteria being evaluated}

We want save the time and resources in our XML transformer, so efficiency is the most element that we consider.
This is not only the speed of reading and writing from the cache, we must also weigh the persistence of the cache to avoid recompiling when the system (application or host) is restarted.

\paragraph{Comparison breakdown}

The first option is storing the cache in-memory, this is the faster and easy way to store cache.

The second option is to create a binary file with the cache.
When we are run our application, we read the cache data in from the cache file into memory.
When the cache is updated it is written back to the original file.

The third option technology is that we create database to handle memory.
This spends time to design and create database.

The fourth option is to use the in-memory key-value store Redis \cite{redis} which is a popular for this kind of task.

\begin{center}
    \begin{tabular}{ | l | p{10cm} |}
    \hline
    Technology & Description  \\ \hline

    Memory &
    \begin{itemize}
      \item Application check data from memory, and put cache into memory.
      \item Retrieving data from memory is the faster way.
    \end{itemize} \\ \hline

    Temporary binary file &
    \begin{itemize}
      \item Application loads binary file when it starts. After we close it, application save binary file in external storage driver.
      \item Loading temporary binary file spend time, so it is slower than memory.
    \end{itemize} \\ \hline

    Database &
    \begin{itemize}
      \item Application access data from database.
      \item It takes time to create and manage a database.
    \end{itemize} \\ \hline

    Redis &
    \begin{itemize}
      \item Objects to be serialized before being cached.
      \item Adding an additional dependency to the project.
      \item Less development time spent daemonizing and designing in-memory caching system.
    \end{itemize} \\ \hline
    \end{tabular}
\end{center}

\paragraph{Discussion}

Storing cache into memory is the most easy way, we just need to allocate memory.
However, the main drawback of this technology is when we close application, all of cache data will be wiped out.
We have to compile file next time when we start running the application.
When we are developing the XZES40-Transformer, we find a tools that KeyList is built in XercesC.
This tools is helpful for managing memory, and it has good data structure.

Creating a binary file can avoid losing cache data, but it spends time to load file into memory when application starts.

Creating database is bad option for XZES40-Transformer application because access database spend resource, and it waste time to search cache data.

Redis is an appealing idea, but because it is an additional dependency, and serializing our parsed documents is not something our client wants our application to do, we are not able to pursue it.
If our application had slightly different requirements this may be a viable option.

\paragraph{Selection}

The best option technology is that storing cache into memory.
Although it will lost data after close application, it save the time, and it the faster way.
We may add a 'backup cache' solution to make this the best of both worlds, restoring from the backup when the system restarts but working mostly in-memory.
And we will using KeyList, because it is built in XercesC.
We can just add API, and easy to control the cache storing.

\input{sections/final-report/tech-review-parallel-processing.tex}
% Section Six: Web API
% Owned by Elijah C. Voigt
\subsubsection{Web API (Elijah C. Voigt)}

\paragraph{Options}

Our web API may be implemented via a C++ web application, a Python or Ruby web application, or an Apache webserver CGI script.
Each of these has pros and cons, and each get the job done at some cost and with some benefits.

\paragraph{Goals for use in design}

XZES40-Transformer will be accessible via a web API.
This can be implemented a few different ways, but all of them must accomplish the same goal of allowing people to use the service over a network.
The three options being evaluated here vary in how they achieve this goal, and so they represent more their technology and less the specific implementation which will be used.

\paragraph{Criteria being evaluated}

The core of this application is related to document transformation.
The more time that is spent on non-document transformation tasks should be kept to a minimum.
Any technology we use to implement our web API should be simple, easy to develop, and easy to maintain.
In short, \textit{keep it simple stupid}.

\paragraph{Comparison breakdown}

Our first option is to use an Apache CGI script, which would be a simple Python, Perl, or Ruby file which calls our C program and returns the results (transformed document or error) to the user, all via an Apache server gateway.
The second options is to write a native web application using a web-app framework like Kore to handle HTTP requests.
The third option is to use a python framework like Flask to handle HTTP requests.
Each of these would be something that calls our document transformation app and exposes it to the outside world.
How we handle that is important to consider.

\begin{table}[H]
  \begin{center}
    \begin{tabular}{ | l | p{10cm} |}
      \hline
      Technology & Description  \\ \hline

      Apache CGI Script \cite{cgi-tutorial} &
      \begin{itemize}
        \item HTTP requests are handled by the Apache web-server.
        \item XZES functionality is called by ``shelling out'' to the program and returning the results.
        \item Apache CGI Script requires a running Apache Server on the host.
      \end{itemize} \\ \hline

      Kore web-app framework \cite{kore-io} \cite{kore-feature} &
      \begin{itemize}
        \item HTTP Requests are handled by the Kore framework.
        \item XZES functionality is called natively with C code.
        \item Kore web-app framework acts as an independent daemon.
      \end{itemize} \\ \hline

      Flask web-app framework \cite{flask-site} &
      \begin{itemize}
        \item HTTP Requests are handled by the Flask framework.
        \item XZES functionality is called either natively or by using \inlinecode{exec} to ``shell out''.
        \item Flask web-app framework can act as an independent daemon or be managed by Apache.
      \end{itemize} \\ \hline

    \end{tabular}
  \end{center}
  \caption{Technology being evaluated for the implementation of the web API.}
\end{table}

\paragraph{Discussion}

The above three technologies are all entirely valid choices for our application; each approach the problem from different angles.

The Apache CGI choice is the Occam's Razor solution relative to the others.
Using the Apache web server we can register a script (written in Python, Perl, Ruby, etc) to accept requests and return a response.
This is simple, maintainable, and easy to create; this is especially appealing if we are only concerned with implementing the API and not fancy features like accessing a database or storing user sessions.
This option also allows us to leverage existing Apache Web-server power like load balancing and simple authentication without needing to write those features ourselves, they're just an Apache configuration file away from being a reality.

Kore is a web-app framework which would allow us to develop our application in C/C++, which has it's pros and cons.
C/C++ is notoriously difficult to write, and harder to write \textit{well}, so it may be a time-sink.
That said, it is nice to have a project which is written entirely in one language as contributors (ourselves and others) do not need to learn multiple languages to contribute to the transformer project.

Flask is another web-app framework, but one which is substantially easier to read and write.
This has the benefit of being easier to maintain than a Kore framework, and we can write exactly the level of complexity we want from our API.
On the other hand we would need to maintain a knowledge base of python, python frameworks, and python dependencies, so this is versatile but ultimately not necessarily easier to maintain than a kore framework.

\paragraph{Selection}

Since we are working with Apache on this project, we want to develop a simple solution to our API problem, and the Apache Web server is a powerful tool we will choose to use this in our design.
We will write a simple CGI script (which calls our C binary) and hook this into an Apache Web server.
We will need to depend on the Apache Web server for our project's package, but this should not be as hard as writing a web app ourselves.
We can also use the server to easily host our Web UI, which is a nice bonus.

% Section Ten: CLI
% Owned by Elijah C. Voigt
\subsubsection{Command Line Interface (Elijah C. Voigt)}

\paragraph{Options}

For the CLI we can develop a native C client, a simple Bash client, or split the difference with a Python client.

\paragraph{Goals for use in design}

Each of these CLI tools must be able to construct an HTTP POST request with two documents, send that request to a server, and parse a response payload.

\paragraph{Criteria being evaluated}

The criteria for this, as it is not part of the core functionality, is simplicity to write, maintain, and use for end-users.

\paragraph{Comparison breakdown}

\begin{table}[H]
  \begin{center}
    \begin{tabular}{ | l | p{10cm} |}
      \hline
      Technology & Description  \\ \hline
      C/C++ \cite{} &
      \begin{itemize}
        \item Keeps the code base exclusively C/C++.
        \item Can be distributed with system package managers.
        \item Difficult to write and maintain.
      \end{itemize}\\ \hline
      Python \cite{} &
      \begin{itemize}
        \item Simple to write and maintain.
        \item Requires few external dependencies.
        \item Can be distributed with the \inlinecode{pip} package manager.
      \end{itemize}\\ \hline
      Bash \cite{} &
      \begin{itemize}
        \item Very simple to write.
        \item Quick to write using existing system tools like \inlinecode{curl} or \inlinecode{wget}.
      \end{itemize}\\ \hline
    \end{tabular}
  \end{center}
  \caption{Technology evaluated for the command-line interface.}
\end{table}

\paragraph{Discussion}

The C/C++ option is not necessary at all.
This is an option which ought to be considered, but ultimately isn't worth pursuing as it will be very complicated to write and maintain, especially when simpler script-based options exist.

Python is a great middle ground.
The language comes with many web-request libraries and provides tools for users to upload their application to the \inlinecode{pip} package manager.
This means we can write a tool which performs well, is easy to maintain, requires few external dependencies, and can be downloaded by a package manager.
It will also be available on all platforms which run python, which includes Unix and Windows systems.

Bash is a viable candidate, especially when other tools like \inlinecode{curl} and \inlinecode{wget} will make quick work of the task.
The downside to this is that we cannot host our CLI on any standard package manager for verifiable distribution.
This forces users to download the CLI from our website directly.
Ultimately this will probably be used as a proof of concept, but it is not a final product.

\paragraph{Selection}

We will start by creating a CLI in Bash for testing purposes, and if time allows we will create a more polished CLI in Python.
Python language offers the best of both worlds in terms of simplicity, maintainability, and ease of use for end users, but for the sake of ``getting something out the door'' we will first create a tool that works in Bash.

% Section Seven: Website
% Owned by Shuai Peng
\subsubsection{Website UI}

The Web Interface will be a simple webpage which calls our web API.
For this reason this section focuses mostly on design technologies and less on HTTP request-handling technologies.

\paragraph{Option}

This document reviews three possible technologies we can use to implement our website .
First is plain-text HTML, CSS and Javascript, second is the Bootstrap front-end framework, and third is the Foundation framework.

\paragraph{Goals for use in design}

XZES40-Transformer will have a web-based user interface, which will also be our main user interface.
In considering which technology to use we will focus on making it conform with modern website design practices.
This will allow us to write an application which is hopefully user friendly and intuitive to use.
The website should also be look good.

\paragraph{Criteria being evaluated}

The most important aspect to consider for this user interface is the appearance.
We want to create web pages that work well on most screen sizes.
So the cost, efficiency, visual appeal, and dynamic screen adjustment are what we consider most important in our technology of choice.

\paragraph{Comparison breakdown}

\begin{itemize}
  \item {
    Cost:
    All of our options are open source and totally free and offer free documentation / tutorials.
  }
  \item {
%    Efficiency: CSS give simple plain-text interface.  We don't actually know what the size is, and what the pixel will be on the screen.  We have to try it, then we know what it looks like on our screen.  However, bootstrap and foundation is responsive design.  Both of them can change the size automatically in different size of screen.Both of them provide templates for creating web pages, but CSS do not provide.
    Efficiency:
    Using plain CSS/Javascript/HTML will perform well on most end-user's web-browsers, however it will be difficult to optimize the website to be responsive and adjust for smaller screen sizes.
    Bootstrap and Foundation were created with responsive design in mind.
    Both of them can change the size automatically in different size of screen.
    Both of them provide templates for creating web pages, but plain CSS do not provide.
  }
  \item {
%     Learning speed: CSS is the basis of HTML style sheet.  It is easy to understand and learning, however it hard to make good design.  Bootstrap and foundation spends time to learn, but after we learn the basically knowledge, bootstrap and foundation will be faster than CSS.
    Learning speed:
    CSS is the basis of HTML style sheet.
    It is easy to understand and learn, however it hard to make it look good.
    Bootstrap and foundation spends time to learn, but after we learn the basically knowledge, bootstrap and foundation will be faster than CSS.
  }
\end{itemize}

\begin{table}[H]
  \begin{center}
      \begin{tabular}{ | l | p{10cm} |}
        \hline
        Technology & Description  \\ \hline
        CSS \cite{CSS_intro}&
        \begin{itemize}
          \item CSS is open source, and it the basis style sheet for HTML.
          \item CSS is not able to easily create a web page that fit in different size of screen.
          \item CSS is easy to learn, but it hard to use to make a good website.
        \end{itemize} \\ \hline

        Bootstrap \cite{boot_intro}&
        \begin{itemize}
          \item Bootstrap is open source project with good forum support.
          \item Bootstrap is efficient because it has responsive deign. It provide many templates.
          \item Bootstrap is easy to learn and use.
        \end{itemize} \\ \hline

        Foundation \cite{foundation_intro}&
        \begin{itemize}
          \item Foundation is open source project.
          \item Foundation is efficiency, and it has responsive deign. It provide many templates.
          \item Foundation is easy to lean and provides free tutorials.
        \end{itemize} \\ \hline
      \end{tabular}
  \end{center}
  \caption{Web frameworks considered for the web-based frontend of XZES40-Transformer}
\end{table}

\paragraph{Discussion}

% The table show us that all of them did similar work. However, they have different advantages and drawback. CSS is basis of HTML style sheet, and it works great, but it hard to create web pages beautiful. CSS is not responsive design. This make us hard to move web page into different size of screen, so we don't want to take CSS as our solution technology. Bootstrap and foundation does the same work. Both of them is open source and free to use. However bootstrap is much more stable and more templates, and there is free instructions in the W3C school. Foundation provides the tutorials, but it need to pay.
The table show us that all of them did similar work however, they have different advantages and drawback.
CSS is basis of HTML style sheet, and it works great, but it hard to create web pages beautiful using \textit{just} CSS and HTML.
This make it hard to move web page into different size of screen, so we don't want to take CSS as our solution technology.

Bootstrap and foundation do similar work, and both are open source and free to use.
However bootstrap is much more stable and provides more templates, and there is free instructions in the W3C school for using it.

\paragraph{Selection}

The best option is the Bootstrap for our project because Bootstrap is open source project and has good tutorials in W3C schools.
Bootstrap is the most popular font-end framework for web design, and it still updated by thousands of people.
During we are developing our code, I think that we should have second choice.
The Second choice is plain-HTML interface.
Althoght plain-HTML dose not look beatiful, it is most straight way to present information to user.
Keeping simple is the best way.

% Section Eight: Debian Package
% Owned by Zixun Lu 
%\subsubsection{Debian Package}
%XZES40-Transformer target the Debian operating system. Our team will upload XZES-40.deb through libraries(Xerces, Xalan) to the Apache website. The users can directly download from the website and run in their Debian operation system. 
%\paragraph{Binary packages}
%Binary packages, which contain executables, configuration files, man/info pages, copyright information, and other documentation. These packages are distributed in a Debian-specific archive format. They are usually characterized by having a '.deb' file extension. Binary packages can be unpacked using the Debian utility dpkg (possibly via a frontend like aptitude); details are given in its manual page.
%\paragraph{Source packages}
%Source packages, which consist of a .dsc file describing the source package (including the names of the following files), a .orig.tar.gz file that contains the original unmodified source in gzip-compressed tar format and usually a .diff.gz file that contains the Debian-specific changes to the original source. The utility dpkg-source packs and unpacks Debian source archives; details are provided in its manual page. (The program apt-get can be used as a frontend for dpkg-source.)
%\paragraph{}
\subsubsection{Debian \& Centos Package}

\paragraph{Options}


\paragraph{Goals for use in design}

Our team will release Debian and Centos packages.
Users can directly download these from the website and use these packages to install XZES40-Transformer on their Debian and Centos operation system.

\paragraph{Criteria being evaluated}

We may use Centos and Debian tools to build our package.
We can use Centos operation system to build Centos packages and use Debian operation system to build Debian packages.
Those tools are free to use.

\paragraph{Comparison breakdown}

\begin{table}[H]
  \begin{center}
    \begin{tabular}{ | l | p{10cm} |}
      \hline
      Technology & Description  \\ \hline

      Centos packaging tools \cite{centos-tool} &
      \begin{itemize}
        \item Use in Centos operate system 
        \item It is esay to use
        \item Free.
      \end{itemize} \\ \hline

      FPM \cite{fpm-home} &
      \begin{itemize}
        \item Translates packages from one format to another.
        \item Allows re-use of other system's packages.
        \item Free.
      \end{itemize} \\ \hline

      Debian packaging tools \cite{debian-tool} &
      \begin{itemize}
        \item Use in Debian operate system
        \item It is easy to use
        \item Free.
      \end{itemize} \\ \hline
    \end{tabular}
  \end{center}
  \caption{Technology evaluated for the Linux system package.}
\end{table}

\paragraph{Discussion}

The above tools are all valuable.
If we were going to just develop a Debian package we may only use the Debian tools, and the same goes for CentOS.
These tools are good at creating packages for those specific platforms, but since we intend to develop tools for mutiple platforms (Linux and BSD) using FPM to create cross-OS packages would be very convenient.

\paragraph{Selection}

In the end we will use FPM to develop our packages becuase it makes life very convenient.

\input{sections/final-report/tech-review-bsd-package.tex}
\input{sections/final-report/tech-review-windows-package.tex}
\subsubsection{Conclusion}

To summarize our findings:

\begin{center}
    \begin{tabular}{ | l | p{5cm} | r | }
    \hline
    Section & Author & Technology Choice\\ \hline
    Research and Benchmarking & Zixun Lu & ? \\ \hline
    XML/XSLT Document Transformation & Shuai Peng & ? \\ \hline
    XML/XSLT Document Parsing & Elijah C. Voigt & Xalan C++ \\ \hline
    XML/XSLT Document Caching & Shuai Peng & ? \\ \hline
    XML/XSLT Document Parallel computation & Zixun Lu & ? \\ \hline
    Web API & Elijah C. Voigt & Apache + Python CGI script \\ \hline
    Web Interface & Shuai Peng & ? \\ \hline
    Linux Package & Zixun Lu & ? \\ \hline
    Command Line Interface& Elijah C. Voigt & Bash proof of concept, Python final product \\ \hline
    Windows Packae & Shuai Peng & ? \\ \hline
    BSD Package & Elijah C. Voigt & FPM package creation toolkit \\ \hline
    \end{tabular}
\end{center}

\clearpage


\subsection{Changes to technologies used}
%%%%%%%%%%%%%%%%%%%%%%%%%%%%%%%%%%%%%%%%%%%%%%%%%%%%%%%%%%%%%%%%%%%%%%%%%%%%%%%%
% Did you change your mind about any technologies?
% What had to change?
%%%%%%%%%%%%%%%%%%%%%%%%%%%%%%%%%%%%%%%%%%%%%%%%%%%%%%%%%%%%%%%%%%%%%%%%%%%%%%%%

\subsubsection{Interface (web) technologies}

In the course of development we did not do a particularly good job of defining how we ought to create the web front-end.
As a result it was thrown together fairly quickly a few weeks before the final deadline.
The tools used in this component were not explicitly researched, but they were well thought out before development.

Specifically we used jQuery to perform asynchronous calls to the Apache CGI script, FileSaver.js and Blob.js (discussed further in the project's Markdown documentation) to download the resulting files to the users computer, and raw HTML5 and CSS3 to format the page.
These technologies worked well and are industry standard tools for small front-end web pages like this one.

\subsubsection{Cache}

The cache ended up using a Key-value library provided to us by our client already in the Xerces/Xalan code-bases.
This was ill-defined early on because we had not yet decided if we were going to write a cache system in-memory ourselves or use a different tool like Redis.
The key-value scheme worked well and was very performed well in testing.

\subsubsection{Parallel/Daemon}

Although we discussed using higher levels of abstraction for creating forking processes.
We ended up just using the system default Linux POSIX threads as this was the technology we were most comfortable with.

\subsubsection{Vagrant}

We expected to use VirtualBox for development but ended up using Vagrant as this was the tool one of the developers was most comfortable with and he was willing to put in extra effort to make this system robust enough for the project's needs.
This ended up helping the team write setup scripts, used by Vagrant to setup a "one command" testing environment, which can be used at a later date for creating a system package.

The Vagrant environment automatically downloads the required operating system (Debian), installs the required packages, copies over the required configuration files (in Apache and systemd), and starts the web server.
Users who want to contribute can run the service locally just need to install Vagrant and run `vagrant up' to get a development environment; further change are automatically copied into the VM and they can compile code in the os by running `vagrant ssh' and running `make' in the VM.

This tool was used for its robust features, simple interface, and cross platform compatibility.
