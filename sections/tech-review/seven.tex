% Section Seven: Website
% Owned by Shuai Peng
\section{Website}
There is 3 different way to create a website pages. First is HTML which is the widely using website language. Second is JavaScript, which is dynamic programming language used for creating website. Third is PHP.

In my opinion, I prefer to take HTML as my solution technology for this project. The first reason is that HTML is easy website language for everyone, and it support other script language, such as CSS and JavaScript. PHP can create dynamic pages, but it work better on server, not for client. The second reason is that we don't need many dynamic information in our website interface. HTML can feedback a link to user via ajax, so user can download new xml file.

\subsection{}
HTML stands for Hyper Text Markup Language. HTML was designed to display data - with focus on how data looks. HTML is a major website language in the Internet. HTML elements are resented by tag. HTML elements are the building blocks of HTML pages.

\subsection{}
PHP (Hypertext Preprocessor) is a server scripting language, and it also is a powerful tool for making dynamic and interactive web Pages. PHP can be embedded into HTML. Here is the PHP usage state, almost 224M sites use this language to make a website.

\subsection{}
JavaScript is the programming language of HTML and the web. However, HTML do display data, it shows static information. JavaScript can improve that function, so we will have dynamic information on the website. JavaScript is prototype-based with first-class functions, making it a multiparadigm language, supporting object-oriented, imperative, and functional programming styles.