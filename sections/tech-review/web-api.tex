% Section Six: Web API
% Owned by Elijah C. Voigt
\section{Web API}

\subsection{Options}

\subsection{Goals for use in design}

XZES40-Transformer will be accessible via a Web API.
This can be implemented a few different ways, but all of them must accomplish the same goal to allow people across the world to use the service over internet protocols.
Three options being evaluated here vary widely in the way they achieve this goal, and so may represent more their technology and less the specific implementation.

\subsection{Criteria being evaluated}

The core of this application is related to document transformation.
To that extent, time spent on non-document transformation tasks should be kept to a minimum.
So for use in our design, any Web API should be simple, easy to develop, and easy to maintain.
In short, \textit{keep it simple stupid}.

\subsection{Comparison breakdown}

The first options is to use an Apache CGI script, which would be a simple Python, Perl, or Ruby file which calls our C program and delivers returns the results to the user, all via Apache.
The second options is to write a native web application using a web-app framework like Kore to handle HTTP requests.
The third option is to use a python framework like Flask to handle HTTP requests.
Each of these would be something that calls our document transformation functionality and exposes it to the outside world, how we handle that is important to consider.

\begin{center}
    \begin{tabular}{ | l | p{10cm} |}
    \hline
    Technology & Description  \\ \hline
    Apache CGI Script & Description \\ \hline
    Kore Web-application framework & Description \\ \hline
    Flask Web-application framework & Description \\ \hline
    \end{tabular}
\end{center}

\subsection{Discussion}

\cite{kore-io}
\cite{kore-feature}
\cite{cgi-tutorial}
\cite{flask-site}

\subsection{Selection}

We will chose to use an Apache CGI script for our web API.
